
% License:
% CC BY-NC-SA 3.0 (http://creativecommons.org/licenses/by-nc-sa/3.0/)
%
%%%%%%%%%%%%%%%%%%%%%%%%%%%%%%%%%%%%%%%%%

%----------------------------------------------------------------------------------------
%	PACKAGES AND OTHER DOCUMENT CONFIGURATIONS
%----------------------------------------------------------------------------------------

\documentclass[paper=a4, fontsize=11pt]{scrartcl} % A4 paper and 11pt font size

\usepackage[T1]{fontenc} % Use 8-bit encoding that has 256 glyphs
\usepackage{fourier} % Use the Adobe Utopia font for the document - comment this line to return to the LaTeX default
\usepackage[english]{babel} % English language/hyphenation
\usepackage{amsmath,amsfonts,amsthm} % Math packages
\usepackage{lipsum} % Used for inserting dummy 'Lorem ipsum' text into the template

\usepackage{caption}
\usepackage{subcaption}
\usepackage{graphicx}

\usepackage{float}

\usepackage{blindtext} %for enumarations

\usepackage[]{hyperref}  %link collor

%talbe layout to the right
%\usepackage[labelfont=bf]{caption}
%\captionsetup[table]{labelsep=space,justification=raggedright,singlelinecheck=off}
%\captionsetup[figure]{labelsep=quad}

\usepackage{sectsty} % Allows customizing section commands
\allsectionsfont{\normalfont\scshape} % Make all sections centered, the default font and small caps

\usepackage{fancyhdr} % Custom headers and footers
\pagestyle{fancyplain} % Makes all pages in the document conform to the custom headers and footers
\fancyhead{} % No page header - if you want one, create it in the same way as the footers below
\fancyfoot[L]{} % Empty left footer
\fancyfoot[C]{} % Empty center footer
\fancyfoot[R]{\thepage} % Page numbering for right footer
\renewcommand{\headrulewidth}{0pt} % Remove header underlines
\renewcommand{\footrulewidth}{0pt} % Remove footer underlines
\setlength{\headheight}{13.6pt} % Customize the height of the header

\numberwithin{equation}{section} % Number equations within sections (i.e. 1.1, 1.2, 2.1, 2.2 instead of 1, 2, 3, 4)
\numberwithin{figure}{section} % Number figures within sections (i.e. 1.1, 1.2, 2.1, 2.2 instead of 1, 2, 3, 4)
\numberwithin{table}{section} % Number tables within sections (i.e. 1.1, 1.2, 2.1, 2.2 instead of 1, 2, 3, 4)

%\setlength\parindent{0pt} % Removes all indentation from paragraphs - comment this line for an assignment with lots of text

\usepackage{listings}

\setlength\parskip{4pt}

%----------------------------------------------------------------------------------------
%	TITLE SECTION
%----------------------------------------------------------------------------------------

\newcommand{\horrule}[1]{\rule{\linewidth}{#1}} % Create horizontal rule command with 1 argument of height

\title{	
\normalfont \normalsize 
\horrule{0.5pt} \\[0.4cm] % Thin top horizontal rule
\huge  DIV DRAC specification version v0.1 \\ % The assignment title
\horrule{2pt} \\[0.5cm] % Thick bottom horizontal rule
}

\author{Victor Soria Pardos} % Your name

\date{10 Jun 2020} % Today's date or a custom date

\begin{document}
%\nocite{*}
\maketitle % Print the title

\newpage
\tableofcontents
\newpage

\section{Version History Table}

\begin{table}[h]
	\centering
	\begin{tabular}{|c|c|c|l|}
		\hline
		Version & Date       & Author     & Note                                \\ \hline
		0.1     & 14/07/2020 & Guillem LP & First draft                         \\ \hline
	\end{tabular}
	\caption{Version History Table}
	\label{tab:version-table}
\end{table}
\newpage

%----------------------------------------------------------------------------------------
%	Section 1
%----------------------------------------------------------------------------------------
\newpage
\section{General purpose of the module}
\textit{Person in Charge; Guillem Lopez Paradis}

The Instruction Fetch Unit module is in charge of controlling the PC register  and making requests to the icache interface in order to pass the instruction to the decoder. It sends requests to the Icache through the Icache Interface and receives an instruction (32 bits) or the corresponding exception. Then, it passes the data, from the Icache, to the decoder.

It is also connected to the Control Logic, which can change the next PC, and also to the branch predictor.


%----------------------------------------------------------------------------------------
%	Section 2
%----------------------------------------------------------------------------------------

\section{Design placement}
\label{chapter2}

Branch prediction modules are meant to be placed inside the fetch\_stage module.

\section{Parameters}
\label{chapter3}

All parameters, enums and types used in this module are defined in drac\_pkg or risc\_pkg.

\section{Interface}
\label{chapter 4}

In this section we describe the interface signals of the module. The structs are not explained and a definition of each parameter can be found in the include package drac\_pkg.sv and riscv\_pkg.sv. 

\begin{table}[h!]
\centering
\begin{tabular}{l|l|l|p{3cm}}
\hline \hline
Signal name & Width (bits) or struct & Input & Description \\
\hline \hline
clk\_i  & 1 & top & clock of the system \\ \hline
rstn\_i & 1 & top & reset of the system \\ \hline
reset\_addr\_i & addr\_t & top & addr to fetch on reset\\ \hline
soft\_rstn\_i & 1 & top & sw reset\\ \hline
resp\_icache\_cpu\_i & resp\_icache\_cpu\_t & icache interface & response from icache \\ \hline
resp\_dcache\_cpu\_i & resp\_dcache\_cpu\_t & dcache interface & response from dcache \\ \hline 
resp\_csr\_cpu\_i & resp\_csr\_cpu\_t & top & response from CSR \\ 
\hline
\end{tabular}
\end{table}

\begin{table}[H]
\centering
\begin{tabular}{l|l|l|p{3cm}}
\hline \hline
Signal name & Width & Output & Description \\
\hline \hline
req\_cpu\_dcache\_o & req\_cpu\_dcache\_t & top & req to dcache \\ 
\hline
req\_cpu\_icache\_o & req\_cpu\_icache\_t & top & req to icache \\ 
\hline
req\_cpu\_csr\_o & req\_cpu\_csr\_t & top & req to CSR \\ 
\hline
\end{tabular}
\end{table}

%input logic             clk_i,
%input logic             rstn_i,
%input addr_t            reset_addr_i,
%input logic             soft_rstn_i,
%input resp_icache_cpu_t  resp_icache_cpu_i,
%input resp_dcache_cpu_t  resp_dcache_cpu_i,
%input resp_csr_cpu_t     resp_csr_cpu_i,

%output req_cpu_dcache_t req_cpu_dcache_o, 
%output req_cpu_icache_t req_cpu_icache_o,
%output req_cpu_csr_t    req_cpu_csr_o

\section{Reset behavior}

Branch predictor has no reset signal. 



\section{What could not happen}

There are some invalid combinations of signals.

The state machine has 4 states: \textit{ResetState}, \textit{Idle}, \textit{MakeRequest}, and \textit{WaitResponse}.
It can not move outside these states.


\section{Behavior}
\label{behavior}

For each output signal, the behavior should be the following:

\begin{itemize}
    \item \textbf{resp\_dcache\_cpu\_o:} this structure is used to send signals to the datapath and it has 8 signals:
    \begin{itemize}
        \item \textbf{lock}: the data cache is busy, and the datapath should be locked until the data cache finishes.
        \item \textbf{ready}: the data cache is ready to send the response.
        \item \textbf{data}: data to serve the load requests.
        \item \textbf{xcpt\_\{ma,pf\}\_\{st,ld\}}: 4 signals to indicate exceptions. \textit{ma} is misaligned, \textit{pf} is DTLB miss, \textit{st} is store, and \textit{ld} is load.
        \item \textbf{addr}: address of the exception.
    \end{itemize}
    \item \textbf{dmem\_req\_valid\_o:} sending a valid request to data cache.
    \item \textbf{dmem\_req\_cmd\_o:} type of the memory access: load, store, load-link, store-conditional, and atomic memory operations.
    \item \textbf{dmem\_req\_addr\_o:} address of the request.
    \item \textbf{dmem\_op\_type\_o:} granularity of the memory access: byte, halfword, and word.
    \item \textbf{dmem\_req\_data\_o:} data to store.
    \item \textbf{dmem\_req\_tag\_o:} tag of the memory access, only for multi-processor.
    \item \textbf{dmem\_req\_invalidate\_lr\_o:} reset load-link/store-conditional transaction.
    \item \textbf{dmem\_req\_kill\_o:} kill actual memory access.
\end{itemize}

The module has a state machine to manage the requests.
The state machine has 4 states:

\begin{itemize}
    \item \textbf{ResetState:} we reach this state when a reset signal arrives, i.e. when \textit{rstn\_i} becomes 0. It reset the state machine, going to \textit{Idle} state, and it cancels the requests to the Data Cache.
    \item \textbf{Idle:} we will stay in this state as long as there are no request from the datapath. If a request arrives from the datapath, we will move to \textit{MakeRequest}.
    \item \textbf{MakeRequest:} if a request arrives when we are in \textit{Idle}, we will move to this state. In this state, we make a request to the Data Cache, and we move immediately to \textit{WaitResponse}.
    \item \textbf{WaitResponse:} we will remain in this state until the Data Cache responds, and then, we will go to \textit{Idle} to wait a new request.
\end{itemize}

If during a request, a signal arrives killing the operation, the state machine will go to \textit{Idle}, and it will cancel the request in flight.



\section{Special cases, corner cases}

\begin{itemize}
  \item The multiplication can overflow
  \item Any combination of inputs that are not defined in the previous section will produce zeros in the outputs.
\end{itemize}


\begin{thebibliography}{9}

\bibitem{longDivision}
Long Division Algorithm \url{https://en.wikipedia.org/wiki/Long_division}, Wikipedia. Last accessed 27 of April of 2020.

\end{thebibliography}

\end{document}
