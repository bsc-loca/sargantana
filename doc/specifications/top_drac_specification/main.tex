
% License:
% CC BY-NC-SA 3.0 (http://creativecommons.org/licenses/by-nc-sa/3.0/)
%
%%%%%%%%%%%%%%%%%%%%%%%%%%%%%%%%%%%%%%%%%

%----------------------------------------------------------------------------------------
%	PACKAGES AND OTHER DOCUMENT CONFIGURATIONS
%----------------------------------------------------------------------------------------

\documentclass[paper=a4, fontsize=11pt]{scrartcl} % A4 paper and 11pt font size

\usepackage[T1]{fontenc} % Use 8-bit encoding that has 256 glyphs
\usepackage{fourier} % Use the Adobe Utopia font for the document - comment this line to return to the LaTeX default
\usepackage[english]{babel} % English language/hyphenation
\usepackage{amsmath,amsfonts,amsthm} % Math packages
\usepackage{lipsum} % Used for inserting dummy 'Lorem ipsum' text into the template

\usepackage{caption}
\usepackage{subcaption}
\usepackage{graphicx}

\usepackage{float}

\usepackage{blindtext} %for enumarations

\usepackage[]{hyperref}  %link collor

%talbe layout to the right
%\usepackage[labelfont=bf]{caption}
%\captionsetup[table]{labelsep=space,justification=raggedright,singlelinecheck=off}
%\captionsetup[figure]{labelsep=quad}

\usepackage{sectsty} % Allows customizing section commands
\allsectionsfont{\centering \normalfont\scshape} % Make all sections centered, the default font and small caps

\usepackage{fancyhdr} % Custom headers and footers
\pagestyle{fancyplain} % Makes all pages in the document conform to the custom headers and footers
\fancyhead{} % No page header - if you want one, create it in the same way as the footers below
\fancyfoot[L]{} % Empty left footer
\fancyfoot[C]{} % Empty center footer
\fancyfoot[R]{\thepage} % Page numbering for right footer
\renewcommand{\headrulewidth}{0pt} % Remove header underlines
\renewcommand{\footrulewidth}{0pt} % Remove footer underlines
\setlength{\headheight}{13.6pt} % Customize the height of the header

\numberwithin{equation}{section} % Number equations within sections (i.e. 1.1, 1.2, 2.1, 2.2 instead of 1, 2, 3, 4)
\numberwithin{figure}{section} % Number figures within sections (i.e. 1.1, 1.2, 2.1, 2.2 instead of 1, 2, 3, 4)
\numberwithin{table}{section} % Number tables within sections (i.e. 1.1, 1.2, 2.1, 2.2 instead of 1, 2, 3, 4)

%\setlength\parindent{0pt} % Removes all indentation from paragraphs - comment this line for an assignment with lots of text


\setlength\parskip{4pt}

%----------------------------------------------------------------------------------------
%	TITLE SECTION
%----------------------------------------------------------------------------------------

\newcommand{\horrule}[1]{\rule{\linewidth}{#1}} % Create horizontal rule command with 1 argument of height

\title{	
\normalfont \normalsize 
\horrule{0.5pt} \\[0.4cm] % Thin top horizontal rule
\huge  Top\_drac specification version v0.1 \\ % The assignment title
\horrule{2pt} \\[0.5cm] % Thick bottom horizontal rule
}

\author{ Guillem Cabo Pitarch} % Your name

\date{20 May 2020} % Today's date or a custom date

\begin{document}
%\nocite{*}
\maketitle % Print the title

\newpage
\tableofcontents

%----------------------------------------------------------------------------------------
%	Section 1
%----------------------------------------------------------------------------------------

\newpage
\section{General purpose of the module}
\textit{Person in Charge; Guillem Lopez Paradis}

The Decoder Unit module is in charge of decoding the instruction fetched previously and outputs all the necessary control signals to properly execute the instruction in the next stages of the pipeline. The input is the instruction fetched in the last cycle and the output are control signals and the immediate generated.

It is also connected to the immediate module that handles the immediate generation. It can also generate a change of PC for some instructions, so it also sends this signal to the control unit.


%----------------------------------------------------------------------------------------
%	Section 2
%----------------------------------------------------------------------------------------

\section{Design placement}
\label{chapter2}

Div\_unit module should be instantiated inside the Exe\_stage module.

Div\_4bits module is meant to be placed inside Div\_unit module. There should be only one instance per core. An all signals of the module are connected inside exe\_stage module.

\section{Parameters}
\label{chapter3}

The instance of this module takes no parameters. Despite that address width is a constant set by the constant REGFILE\_WIDTH. Register and port width is set by reg64\_t and bus64\_t types.
REGFILE\_WIDTH, reg64\_t and bus64\_t are defined in drac\_pkg.


\section{Interface}
\label{chapter 4}

In this section we describe the interface signals of the module.

\begin{table}[H]
\centering
\begin{tabular}{l|l|l|p{3cm}}
\hline
Signal name & Width (bits) or struct & Input & Description \\
\hline
clk\_i  & 1  & datapath & General input clock \\
\hline
rstn\_i & 1  & datapath & General neg reset \\
\hline
reset\_addr\_i & 40 & datapath & Reset addr from the SoC \\ 
\hline
stall\_i & 1 & control unit & Stalls the fetch (CU) \\
\hline
cu\_if\_i & struct cu\_if\_t & control unit & selects next pc (CU) \\
\hline
invalidate\_icache\_i  & 1 & control unit & Signal to Icache interface to invalidate the icache (e.g. fence) \\
\hline
invalidate\_buffer\_i  & 1 & control unit & Signal to Icache interface to invalidate the instruction buffer (e.g. fence) \\
\hline
pc\_jump\_i  & 64  & datapath & next pc addr coming from commit, ecall, decode \\
\hline
resp\_icache\_cpu\_i  &  resp\_icache\_cpu\_t & icache\_interface & Response from icache interface (instruction) \\ 
\hline
exe\_if\_branch\_pred\_i & exe\_stage & exe\_if\_branch\_pred\_t & Signal for branch predictor coming from exe stage (correct info of the branch) \\
\hline
retry\_fetch\_i & 1 & datapath & Retry fetch when there is an evec \\
\hline
\end{tabular}
\end{table}

\begin{table}[H]
\centering
\begin{tabular}{l|l|l|p{3cm}}
\hline \hline
Signal name & Width & Output & Description \\
\hline \hline
req\_cpu\_icache\_o & req\_cpu\_icache\_t struct & if\_stage -> if\_interface & Request packet going to the Icache Interface\\ \hline
if\_id\_stage\_t & if\_id\_stage\_t struct & if\_stage -> datapath (decoder) & Fetch data output\\
\hline
\end{tabular}
\end{table}

\section{Reset behavior}

This module only contains continuous assignments and wires among modules, therefore there is no special reset condition. 

All in all is important to highlight that the input reset signal does not follow the name convention of adding \_n for active low signals. Despite that RST and SOFT\_RST signals are active LOW, and they do not need to be inverted when interfacing with other modules. 







\section{What could not happen}

\subsection{What could not happen in exe\_stage module}

In case that \textbf{from\_rr\_i.instr.ex.valid} is equal to one, the following conditions are always true:
\begin{itemize}
  \item to\_wb\_o.ex.valid == 1
  \item to\_wb\_o.ex.origin == from\_rr\_i.instr.ex.origin
  \item to\_wb\_o.ex.cause == from\_rr\_i.instr.ex.cause
\end{itemize}



\subsection{What could not happen in ALU module}

In case that instr\_type is one of:
\begin{itemize}
  \item ADD
  \item ADDW
  \item SUB
  \item SUBW
  \item SLL
  \item SLLW
  \item SLT
  \item SLTU
  \item XOR
  \item SRL
  \item SRLW
  \item SRA
  \item SRAW
  \item OR
  \item AND
\end{itemize}

Signals data\_rs1\_i and data\_rs2\_i must have known values. This means that signals must be either 0 or 1, but not meta-stable values.
And in case instr\_type is none of the above result\_o should always be 0.

\subsection{What could not happen in Branch Unit module}

In case that \textbf{instr\_type\_i} is one of:
\begin{itemize}
  \item JAL
  \item JALR
  \item BEQ
  \item BNE
  \item BLT
  \item BGE
  \item BLTU
  \item BGEU
\end{itemize}

Signals pc\_i, imm\_i, data\_rs1\_i and data\_rs2\_i must have known values. This means that signals must be either 0 or 1, but not meta-stable values. And in case \textbf{instr\_type} is none of the above \textbf{taken\_o} should always be \emph{PRED\_NOT\_TAKEN}.

In case that \textbf{instr\_type\_i} is JAL, \textbf{taken\_o} is always \emph{PRED\_NOT\_TAKEN}, because the jump was done at decode stage.

In case that \textbf{instr\_type\_i} is JALR, \textbf{taken\_o} is always \emph{PRED\_TAKEN}.

In case that \textbf{taken\_o} is \emph{PRED\_NOT\_TAKEN}, \textbf{result\_o} is always \textbf{pc\_i} plus \emph{0x4}.

The two lower bits of \textbf{result\_o} and \textbf{link\_pc\_o} must be always 0. 
    

\section{Behavior}

\subsection{Description}

\subsubsection{General case}
The general case is that in a given cycle the decoder receives a valid instruction in raw format from the ICache, with a PC, branch prediction and an exception struct and is decoded. The output can be valid/invalid and with/without exception. By default, it is followed the RISCV ISA in a strict manner to mark as illegal instruction if for example, there is a combination of opcode and funct7 code that is not legal. 

Following, an explanation for the most relevant bits is given to later explain some interesting cases such as the JAL, system instructions, illegal instructions and atomics among others.

\subsubsection{Most relevant bits}
\begin{table}[H]
	\centering
	\begin{tabular}{l|l|p{10cm}}
		\hline \hline
		Signal name & Width & Description \\
		\hline \hline
		use\_imm & 1 & Specifies an instruction that operates with immediate values \\
		\hline
		use\_pc & 1 & Specifies an instruction that uses the PC, like a branch\\
		\hline
		op\_32 & 1 &  Specifies an instruction that operates on 32 bits instead of 64 \\
		\hline
		change\_pc\_ena & 1 & Control bit that enables the instruction to change the PC \\
		\hline
		regfile\_we & 1 & Control bit that enables the instruction to write to the register file  \\
		\hline
		instr\_type & 7  & Control bits that specify the internal instruction type to be fed to the ALU, Branch Unit or DCache interface \\
		\hline
		result & 64 & In this field, the immediate is placed. Afterwards, it will be used as the result of the given instruction  \\
		\hline
		signed\_op & 1 & Bit used to mark as a signed operation in the case of some divisions and remainder operations  \\
		\hline
		mem\_size & 2 & Bits that specify the memory operation to be sent to lowrisc DCache \\
		\hline
		stall\_csr\_fence & 1 & Control bit that tells the instruction whether it is a csr or a fence \\
		\hline
	\end{tabular}
\end{table}

\subsection{JAL}
The JAL instruction is the only one that can change the PC at this stage of the pipeline. It is a non-conditional branch that if valid, will change the PC and make it jump to the address calculated at this point. For this reason, and following the RISCV ISA, it can also generate a misaligned exception that needs to be reported by the decoder. If this happens, the instruction will not jump and it will continue the pipeline with the exception to be treated at commit stage. 


\subsection{Illegal Instruction}
Following the RISCV ISA, in the 32 bits of the instruction, only a subset of opcodes are valid and a lot of instructions have some restrictions in the opcodes to be used, or some auxiliary function codes that only some values are valid. If any of the last restrictions are violated by the incoming instruction, the decoder will report an illegal instruction.


\subsection{Atomics}
In the atomics extension, there are some bits of the instruction that we are currently ignoring. Bits such as \textit{aq} and \textit{rl} which specifies additional memory ordering constraints.

From the ISA: \textit{If both bits are clear, no additional ordering constraints are imposed on the atomic memory operation. If only the aq bit is set, the atomic memory operation is treated as an acquire access, i.e., no following memory operations on this RISC-V hart can be observed to take place before the	acquire memory operation. If only the rl bit is set, the atomic memory operation is treated as a release access, i.e., the release memory operation cannot be observed to take place before any earlier memory operations on this RISC-V hart. If both the aq and rl bits are set, the atomic memory operation is sequentially consistent and cannot be observed to happen before any earlier memory operations or after any later memory operations in the same RISC-V hart and to the same address	domain.}

\subsection{System Instructions}
The current system ISA supported is 1.7, but it is not completely tested and a new ISA is available. Hence, although all instructions are there, there is a limited support and the plan is to change to the latest ISA. Also, ZicFenceI extension is not supported at the moment.

\subsection{FENCE}
FENCE instruction does not take into account the \textit{fm/pred/succ} bits.


\subsection{Examples}

Some examples are given, only showing the relevant signals. The cases are:

\begin{itemize}
	\item Cycles 1: Invalid input --> invalid output. 
	\item Cycles 2: Exception input --> exception output and not decoded.
	\item Cycles 3: Regular case, valid input, not illegal instruction, and signed operation.
	\item Cycles 4: Regular case again, but it has the power to change the pc.
	\item Cycles 5: Case with illegal instruction, hence exception valid to 1. 
	\item Cycles 6: Case of the fence with the special signal to 1.
\end{itemize}

In the next figure we can see all the cases mentioned before.

\begin{figure}[H]
	\centering
	\includegraphics[width=\textwidth]{Figure/full_example.png}
	\caption{Example of expected behavior of the decoder.}
\end{figure}





\section{Special cases, corner cases}

\subsection{Div\_unit}

\begin{itemize}
  \item When \emph{divsr\_i} is zero
  \item Set \emph{int\_32\_i} to 1, while \emph{dvnd\_i} and \emph{dvsr\_i} are set to 64 bit variables.
  \item Examples in which  \emph{dvnd\_i} and \emph{dvsr\_i} have different sign for 32 and 64 bit.
  \item Examples in which \emph{signed\_op\_i} is set and \emph{dvnd\_i} and \emph{dvsr\_i} have negative sign.
\end{itemize}



\end{document}
